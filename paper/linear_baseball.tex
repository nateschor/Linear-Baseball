\documentclass{article}
\usepackage{setspace}
\usepackage[utf8]{inputenc}
\usepackage{natbib}
\usepackage{url}
\usepackage{indentfirst} 
\usepackage{hyperref}
\usepackage{xcolor}
\usepackage{float}

\usepackage{booktabs}
\usepackage{longtable}
\usepackage{array}
\usepackage{multirow}
\usepackage{wrapfig}
\usepackage{float}
\usepackage{colortbl}
\usepackage{pdflscape}
\usepackage{tabu}
\usepackage{threeparttable}
\usepackage{threeparttablex}
\usepackage[normalem]{ulem}
\usepackage[utf8]{inputenc}
\usepackage{makecell}
\usepackage{xcolor}

\usepackage{graphicx}
\graphicspath{ {figures/} }

\title{Constructing Optimal MLB Teams with Linear Programming}
\author{Nathan Schor}
\date{August 28, 2022}

\doublespacing
\begin{document}

\maketitle
\begin{singlespace}
\tableofcontents
\end{singlespace}

\newpage

\section{Introduction}

The purpose of this research is to investigate the relationship between salary and team performance in Major League Baseball (MLB). We do this by looking at actual outcomes for MLB teams in 2021 and compare those results to optimal rosters using linear programming. Our goal is to analyze the gap between current team performance and their potential optimal performance. 

We begin with 2021 salary data for each MLB team. This data is supplemented it with each player's 2021 salary, team, position, and JEFFBAGWELL (our performance metric, abbreviated JB). Next, we seek to maximize JEFFBAGWELL for each of the 30 teams subject to salary and player position constraints, and visualize the results. 

\section{Literature Review}

The relationship between spending and performance is common in both the public baseball discourse and in the baseball community (SABR). The book \emph{Moneyball} began the baseball revolution. One example of using linear programming in baseball is \textbf{R LINEAR PROGRAMMING LINK}. Other examples of linear programming in baseball are \textbf{FIND MORE EXAMPLES}. 

\section{Methodology/Data}

\subsection{Data Cleaning}

To clean the player dataset, we first assign positions to each player. A player's position is characterized as the position they played most frequently during the season (if they played two positions an equal amount of times, their position was randomly assigned to one of the two positions). Player's with a \$0 salary or a missing salary are removed. A pitcher is classified as a Starting Pitcher (SP) if $\geq$ 50\% of their appearances were as a starter, and otherwise are classified as a Relief Pitcher (RP). Furthermore, First Basemen (1B), Second Basemen (2B), Third Basemen (3B), and Shortstops (SS) are classified as Infielders (IF). Left Fielders (LF), Center Fielders (CF), and Right Fielders (RF) are classified as Outfielders (OF). Catchers (C) and Designated Hitters (DH) are left as their own individual categories. 

\subsection{Solving the Optimization Problem}

The decision variables ($x_{i}$) are which MLB players will be selected for each team. $x_{i}$ is a binary variable that equal 1 if player \emph{i} is chosen, and 0 if they are not. The objective function is to maximize the JB for each of the 30 teams (T):
\begin{equation}
\sum_{i = 1}^{N} x_{i} * JB_{i}
\end{equation} where \emph{N} is the total number of eligible players in 2021.

subject to the following constraints:

\begin{equation}
\sum_{i = 1}^{N} x_{i} = 25
\end{equation}
\begin{equation}
\forall x \in SP:  \sum_{i = 1}^{N} x_{i} = 5
\end{equation}
\begin{equation}
\forall x \in RP:  \sum_{i = 1}^{N} x_{i} = 7 
\end{equation}
\begin{equation} 
\forall x \in CF:  \sum_{i = 1}^{N} x_{i} \geq 1
\end{equation}
\begin{equation} 
\forall x \in RF:  \sum_{i = 1}^{N} x_{i} \geq 1 
\end{equation}
\begin{equation} 
\forall x \in LF:  \sum_{i = 1}^{N} x_{i} \geq 1 
\end{equation}
\begin{equation} 
\forall x \in 2B:  \sum_{i = 1}^{N} x_{i} \geq 1 
\end{equation}
\begin{equation} 
\forall x \in 3B:  \sum_{i = 1}^{N} x_{i} \geq 1
\end{equation} 
\begin{equation} 
\forall x \in 1B:  \sum_{i = 1}^{N} x_{i} \geq 1 
\end{equation}
\begin{equation} 
\forall x \in SS:  \sum_{i = 1}^{N} x_{i} \geq 1 
\end{equation}
\begin{equation} 
\forall x \in C:  \sum_{i = 1}^{N} x_{i} = 2 
\end{equation}
\begin{equation}
\forall x \in DH:  \sum_{i = 1}^{N} x_{i} = 1 
\end{equation}
\begin{equation} 
\forall x \in IF:  \sum_{i = 1}^{N} x_{i} = 5 
\end{equation}
\begin{equation} 
\forall x \in OF:  \sum_{i = 1}^{N} x_{i} = 5
\end{equation}
\begin{equation}
\sum_{i = 1}^{N} x_{i} * x_{salary} \leq T_{salary}
\end{equation}

Equation (2) constrains each team to have exactly 25 players. Equations (3)-(15) stipulate the number of players at each position and the total number of players allowed for grouped positions. (16) requires that each team spend no more on players than they did in the actual 2021 season. 


%\begin{table}
%\caption{Number of Players who are on both the Optimal and Actual Teams}
%\label{tab:optimal_and_actual}
\begin{table}
\centering
\begin{tabular}{l|r}
\hline
Team & Number of Players on both Actual and Optimal Team\\
\hline
LAD & 3\\
\hline
WSN & 3\\
\hline
MIL & 2\\
\hline
SFG & 2\\
\hline
TBR & 2\\
\hline
ATL & 1\\
\hline
BOS & 1\\
\hline
CHC & 1\\
\hline
CIN & 1\\
\hline
COL & 1\\
\hline
HOU & 1\\
\hline
LAA & 1\\
\hline
MIN & 1\\
\hline
NYY & 1\\
\hline
OAK & 1\\
\hline
PHI & 1\\
\hline
PIT & 1\\
\hline
SDP & 1\\
\hline
TOR & 1\\
\hline
CHW & 0\\
\hline
NYM & 0\\
\hline
KCR & 0\\
\hline
STL & 0\\
\hline
CLE & 0\\
\hline
MIA & 0\\
\hline
BAL & 0\\
\hline
ARI & 0\\
\hline
SEA & 0\\
\hline
TEX & 0\\
\hline
DET & 0\\
\hline
\end{tabular}
\end{table}

%\end{table}

\begin{figure}[h]
\caption{Boxplot of Salary (for Players with Salary $> 0$) by Position}
\label{fig:salary_position_boxplot}
\centering
\includegraphics[width=0.7\paperwidth, scale=1.25]{salary_position_boxplots.png}
\end{figure}

\begin{figure}[h]
\caption{Boxplot of JEFFBAGWELL by Position}
\label{fig:salary_war_boxplot}
\centering
\includegraphics[width=0.7\paperwidth, scale=1.25]{war_position_boxplots.png}
\end{figure}

\section{Computational Experiment and Results}



\begin{figure}[h]
\caption{Scatterplots of a Team's Total JEFFBAGWELL vs. Total Dollars Spent for the Actual Team (Left)and Optimal Team (Right). The Red and Blue lines are constructed using LOESS.} 
\label{fig:cowplot}
\centering
\includegraphics[width=0.7\paperwidth, scale=1.25]{bwar_salary_scatter_cowplot.png}
\end{figure}


%\begin{table}
%\caption{Five rows from the \emph{Hamilton} dataset.}
%\label{tab:example}
%\input{tables/example_raw_data.tex}
%\end{table}

\section{Discussion and Conclusions}

%\begin{figure}[h]
%    \caption{Ontology for \emph{Hamilton}. \label{fig:ontology}}
%    \centering
%    \includegraphics[width=0.7\paperwidth, scale=1.25]{ontology.png}
%\end{figure}

%\bibliography{references}
%\bibliographystyle{apalike}

\section{Appendix}

\begin{figure}[h]
\caption{Histogram of Salary for Players with Salary $> 0$}
\label{fig:salary_hist}
\centering
\includegraphics[width=0.7\paperwidth, scale=1.25]{salary_hist.png}
\end{figure}

\begin{figure}[h]
\caption{Histogram of JEFFBAGWELL}
\label{fig:bwar_hist}
\centering
\includegraphics[width=0.7\paperwidth, scale=1.25]{war_hist.png}
\end{figure}

\begin{table}

\caption{Number of teams (max of 30) selecting a given player for their optimal roster}
\centering
\fontsize{7}{9}\selectfont
\begin{tabular}[t]{|>{}c|>{}c|}
\hline
Player & Teams Picked\\
\hline
\cellcolor{gray!6}{Brandon Woodruff} & \cellcolor{gray!6}{30}\\
\hline
Fernando Tatis Jr. & 30\\
\hline
\cellcolor{gray!6}{Mike Zunino} & \cellcolor{gray!6}{30}\\
\hline
Shohei Ohtani & 30\\
\hline
\cellcolor{gray!6}{Walker Buehler} & \cellcolor{gray!6}{30}\\
\hline
Andrew Chafin & 29\\
\hline
\cellcolor{gray!6}{Carlos Rodon} & \cellcolor{gray!6}{29}\\
\hline
Juan Soto & 29\\
\hline
\cellcolor{gray!6}{Matt Olson} & \cellcolor{gray!6}{28}\\
\hline
Jesse Winker & 27\\
\hline
\cellcolor{gray!6}{Jose Ramirez} & \cellcolor{gray!6}{26}\\
\hline
Aaron Loup & 25\\
\hline
\cellcolor{gray!6}{Carlos Correa} & \cellcolor{gray!6}{25}\\
\hline
Jacob Stallings & 25\\
\hline
\cellcolor{gray!6}{Aaron Judge} & \cellcolor{gray!6}{24}\\
\hline
Josh Hader & 23\\
\hline
\cellcolor{gray!6}{Kendall Graveman} & \cellcolor{gray!6}{23}\\
\hline
Enrique Hernandez & 22\\
\hline
\cellcolor{gray!6}{Chad Green} & \cellcolor{gray!6}{19}\\
\hline
Robbie Ray & 17\\
\hline
\cellcolor{gray!6}{Liam Hendriks} & \cellcolor{gray!6}{16}\\
\hline
Ryan Tepera & 16\\
\hline
\cellcolor{gray!6}{Brandon Lowe} & \cellcolor{gray!6}{15}\\
\hline
Marcus Semien & 15\\
\hline
\cellcolor{gray!6}{Zack Wheeler} & \cellcolor{gray!6}{14}\\
\hline
Blake Treinen & 12\\
\hline
\cellcolor{gray!6}{Freddy Peralta} & \cellcolor{gray!6}{12}\\
\hline
Adam Cimber & 11\\
\hline
\cellcolor{gray!6}{Julio Urias} & \cellcolor{gray!6}{11}\\
\hline
Brad Boxberger & 10\\
\hline
\cellcolor{gray!6}{Harrison Bader} & \cellcolor{gray!6}{10}\\
\hline
Joey Gallo & 9\\
\hline
\cellcolor{gray!6}{Bryce Harper} & \cellcolor{gray!6}{8}\\
\hline
Luke Jackson & 6\\
\hline
\cellcolor{gray!6}{Teoscar Hernandez} & \cellcolor{gray!6}{6}\\
\hline
Tyler Mahle & 5\\
\hline
\cellcolor{gray!6}{C.J. Cron} & \cellcolor{gray!6}{4}\\
\hline
Caleb Thielbar & 4\\
\hline
\cellcolor{gray!6}{JT Chargois} & \cellcolor{gray!6}{4}\\
\hline
Luis Robert & 4\\
\hline
\cellcolor{gray!6}{Mark Melancon} & \cellcolor{gray!6}{4}\\
\hline
Ryan Pressly & 4\\
\hline
\cellcolor{gray!6}{Salvador Perez} & \cellcolor{gray!6}{4}\\
\hline
Byron Buxton & 3\\
\hline
\cellcolor{gray!6}{Jeimer Candelario} & \cellcolor{gray!6}{3}\\
\hline
Kyle Schwarber & 2\\
\hline
\cellcolor{gray!6}{Michael A. Taylor} & \cellcolor{gray!6}{2}\\
\hline
Michael Fulmer & 2\\
\hline
\cellcolor{gray!6}{Starling Marte} & \cellcolor{gray!6}{2}\\
\hline
Tony Kemp & 2\\
\hline
\cellcolor{gray!6}{Adam Engel} & \cellcolor{gray!6}{1}\\
\hline
AJ Pollock & 1\\
\hline
\cellcolor{gray!6}{Buster Posey} & \cellcolor{gray!6}{1}\\
\hline
Craig Kimbrel & 1\\
\hline
\cellcolor{gray!6}{Frankie Montas} & \cellcolor{gray!6}{1}\\
\hline
Joey Wendle & 1\\
\hline
\cellcolor{gray!6}{Jorge Polanco} & \cellcolor{gray!6}{1}\\
\hline
Luis Cessa & 1\\
\hline
\cellcolor{gray!6}{Max Scherzer} & \cellcolor{gray!6}{1}\\
\hline
\end{tabular}
\end{table}


\end{document}